\documentclass{article}

\usepackage{amsmath, amssymb}

\author{Guillaume Labranche (260585371)}
\title{Assignment \#2\\Numerical Computing (COMP 350)}
\date{due on 30 September 2015}

\newcommand{\R}{\mathbb{R}}
\newcommand{\F}{\mathbb{F}}

\begin{document}

\maketitle
 
\begin{enumerate}%[(a)]

\item $x_{single} = 41$ and $x_{double} = 49$

\item Because this sequence initally rises past $N_{max}$ ($3.402823466 \cdot 10^{38}$) before converging to $0$. So at some $n$, $x_n = \infty$ and since the program depends on $x_n$ to compute $x_{n+1}$, once it hits such $x_n = \infty$ it will always stay at $\infty$. \\
The sequence rises to high values before coming back down because $100^n$ is much higher than $n!$ for at least the first 100 values of $n$.

\end{enumerate}
\end{document}