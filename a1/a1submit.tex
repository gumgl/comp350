\documentclass{article}

\usepackage{amsmath, amssymb}

\author{Guillaume Labranche (260585371)}
\title{Assignment \#1\\Numerical Computing (COMP 350)}
\date{due on 23 September 2015}

\newcommand{\R}{\mathbb{R}}
\newcommand{\F}{\mathbb{F}}

\begin{document}

\maketitle
 
\begin{enumerate}%[(a)]
 \item \textbf{No}. Let $x$ be a real number with finite binary representation and $x_i$ be the i\textsuperscript{th} bit after the point. The integer part can be represented in finite decimal form in all cases. For all $x_i$ in the fractional part, if the digit is $1$, its value within $x$ is $1/(2^i)$ and that in decimal form is finite,  so adding all $x_i$'s leads to a finite decimal representation.
 \item $\boldsymbol{-23}$
 \item 
   \begin{enumerate}
     \item $(2^4)/2-1=16/2-1=8-1=\boldsymbol{7}$
     \item Normalized nonnegative range:
     
\begin{tabular}{ r|l|l|l|l| }
\multicolumn{1}{r}{}
 & \multicolumn{1}{c}{Sign}
 & \multicolumn{1}{c}{Exponent}
 & \multicolumn{1}{c}{Fraction}
 & \multicolumn{1}{c}{Decimal Value} \\
\cline{2-5}
$N_{min}$ & $1$ & $(0001)_2=1$ & $(00000)_2$ & $(1.00000)_2 \cdot 2^{-6}=\boldsymbol{(1/64)_{10}}$
 \\
\cline{2-5}
$N_{max}$ & $1$ & $(1110)_2=14$ & $(11111)_2$ & $(1.11111)_2 \cdot 2^7=\boldsymbol{(252)_{10}}$
 \\
\cline{2-5}
\end{tabular}\\

	\item Subnormal nonnegative range:
	
\begin{tabular}{ r|l|l|l|l| }
\multicolumn{1}{r}{}
 & \multicolumn{1}{c}{Sign}
 & \multicolumn{1}{c}{Exponent}
 & \multicolumn{1}{c}{Fraction}
 & \multicolumn{1}{c}{Decimal Value} \\
\cline{2-5}
Smallest & $1$ & $(0000)_2=1$ & $(00001)_2$ & $(0.00001)_2 \cdot 2^{-6}=\boldsymbol{2^{-11}}$
 \\
\cline{2-5}
Largest & $1$ & $(0000)_2=15$ & $(11111)_2$ & $(0.11111)_2 \cdot 2^{-6}=\boldsymbol{31/2^{-11}}$
 \\
\cline{2-5}
\end{tabular}
	
	\item $\epsilon = \boldsymbol{2^{-5}}$
	\item \textbf{10.75} and \textbf{11.15}
	
\begin{tabular}{ |l|l|l|l| }
   \multicolumn{1}{c}{Sign}
 & \multicolumn{1}{c}{Exponent}
 & \multicolumn{1}{c}{Fraction}
 & \multicolumn{1}{c}{Decimal Value} \\
\cline{1-4}
$1$ & $(1010)_2=10$ & $(01011)_2$ & $(1.01011)_2 \cdot 2^3=10.75$
 \\
\cline{1-4}
$1$ & $(1010)_2=10$ & $(01100)_2$ & $(1.01100)_2 \cdot 2^3=11.0$
 \\
\cline{1-4}
$1$ & $(1010)_2=10$ & $(01101)_2$ & $(1.01101)_2 \cdot 2^3=11.25$
 \\
\cline{1-4}
\end{tabular}
	\item \begin{align*} 
	x   &=-(1.0110101)_2 \cdot 2^0 \\
	x_+ &= -(1.01101)_2 \cdot 2^0 \\
	x_- &= -(1.01110)_2 \cdot 2^0
	\end{align*}
	\begin{itemize}
	  \item Round down: $x_- = -(1.01110)_2 \cdot 2^0$
	  \item Round up: $x_+ = -(1.01101)_2 \cdot 2^0$
	  \item Round towards zero: $x_+ = -(1.01101)_2 \cdot 2^0$
	  \item Round to nearest: $x_+ = -(1.01101)_2 \cdot 2^0$
	\end{itemize}

   \end{enumerate}

\item \begin{enumerate}
\item \textbf{True}. When adding a number $x$ to itself, we are doubling its value ($2x$). In binary representation, this comes down to shifting the decimal point 1 position to the right, so the significant stays the same but $E$ is increased by $1$ when $x$ is in the form $(b_0 . b_1 b_2 \dotso b_{23})_2 \cdot 2^E$.
\begin{align*}
x \oplus x &= 2x \\
\mbox{round}(x + x) &= 2x \\
\mbox{round}(2x) &= 2x \\
\mbox{round}(2 \cdot (b_0 . b_1 b_2 \dotso b_{23})_2 \cdot 2^E) &= 2 \cdot (b_0 . b_1 b_2 \dotso b_{23})_2 \cdot 2^E \\
\mbox{round}((b_0 . b_1 b_2 \dotso b_{23})_2 \cdot 2^{E+1}) &= (b_0 . b_1 b_2 \dotso b_{23})_2 \cdot 2^{E+1}
\end{align*}
\item \textbf{False}. Counter-example (rounding mode is \textbf{round down}):
\begin{align*}
x&= & &(  &  1&.00000000000000000000000|  &  &)_2 \cdot 2^0 \\
y&= & -&(  &  0&.11111111111111111111111|1  &  &)_2 \cdot 2^0 \\
x-y&= & &(  &  0&.00000000000000000000000|1  &  &)_2 \cdot 2^0 \\
\boldsymbol{x \ominus y} &= & &(  &  0&.00000000000000000000000|  &  &)_2 \cdot 2^0\\
y-x&= & -&(  &  0&.00000000000000000000000|1  &  &)_2 \cdot 2^0 \\
y \ominus x &= & -&(  &  0&.00000000000000000000001|  &  &)_2 \cdot 2^0 \\
\boldsymbol{-(y \ominus x)} &= & &(  &  0&.00000000000000000000001|  &  &)_2 \cdot 2^0
\end{align*}
\end{enumerate}
\item \begin{itemize}
\item $\infty / 0 = \boldsymbol{\infty}$ (because $a/0=\infty$)
\item $\infty / (- \infty) = \mbox{\textbf{NaN}}$ (any operation involving NaN results in NaN)
\item $1^{\mbox{NaN}} = \mbox{\textbf{NaN}}$ (same reason)
\item $-0/\mbox{NaN} = \mbox{\textbf{NaN}}$ (same reason)

\end{itemize}

\end{enumerate}
\end{document}