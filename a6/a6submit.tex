\documentclass{article}

\usepackage{mathtools}
\usepackage[margin=0.7in]{geometry}
\usepackage{float}
\usepackage{graphicx}
\usepackage{epstopdf}
\usepackage{xfrac}
\usepackage{hyperref}
\usepackage{xcolor}
\hypersetup{
    colorlinks,
    linkcolor={red!50!black},
    citecolor={blue!50!black},
    urlcolor={blue!80!black}
}
\restylefloat{table}

\author{Guillaume Labranche (260585371)}
\title{Assignment \#6\\Numerical Computing (COMP 350)}
\date{due on 30 November 2015}

\newcommand{\R}{\mathbb{R}}
\newcommand{\F}{\mathbb{F}}

\begin{document}

\maketitle

\begin{enumerate}
\item 
See \texttt{q1.m} for the script displaying the results.
\begin{enumerate}
\item erf(3) was computed to be  0.99997550396668711 in 17 function evaluations ($2^4=16$ subintervals).

See function \texttt{erf\_rtr.m} and \texttt{rtr.m} for the source code.

\item erf(3) was computed to be 0.99997747287883731 in 41 function evaluations. I was able to save function evaluations by passing $f(a)$, $f(b)$ and $f(c)$ to the next iteration step.

See function \texttt{erf\_asm.m} and \texttt{asm.m} for the source code.
\end{enumerate}

\end{enumerate}
\end{document}